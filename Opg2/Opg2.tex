\section{Den kinesiske restklassesætning}

I opgave 2 \textbf{a)} benytter vi tankegangen bag den kinesiske restklassesætning, men ikke sætningen direkte som den er formuleret i noten. I opgave 2 \textbf{b)} benytter vi sætningen med notens notation.

\subsection*{a)}

Vi ønsker at finde løsningsmængden til det givne (kongruens)ligningssystem:
\begin{equation}\label{eq:2a1}
x \equiv 2 \; \; (\text{mod} \; 101)
\end{equation}
\begin{equation}\label{eq:2a2}
x \equiv 1 \; \; (\text{mod} \; 17 )
\end{equation}

Vi genkender de givne modulo 101 og 17 som primtal, og har derfor sfd(101,17) = 1. Der er altså med garanti en løsningen til ligningssystemet ifølge den kinesiske restklassesætning. 

Vi omskriver kongruensligningerne til følgende ligninger, hvor $c,d \in \mathds{Z}$:

\begin{equation}
x = 2 + 101c
\end{equation}
\begin{equation}\label{eq:xd}
 x = 1+17d
\end{equation}

Ligningerne kan omskrives til følgende:

\begin{equation}
2 + 101c = 1 + 17d
\end{equation}
\begin{equation}
\Rightarrow  17d = 1 + 101c
\end{equation}

Som igen kan omskrives til en kongruensligning:
\begin{gather}\label{eq:17d}
    17 d \equiv 1 \; \; (\text{mod} \; 101)
\end{gather}

Vi ønsker nu at finde et heltal $R$ der opfylder $17R \equiv 1 \; \; (\text{mod} \; 101)$, dvs. vi finder den multiplikativ inverse til 17, modulo 101. Dette gøres ved Euklids udvidede algoritme: 

\begin{table}[H]
    \centering
    \begin{tabular}{c|c|c|c}
    $k$ & $r_k$ & $s$  & $R$  \\ \hline
    1& 101   & 1  & 0  \\
    2& 17   & 0 & 1  \\
    3& 16   & 1  & $-$5 \\
    4& 1    & $-$1 & 6 \\
    5 & 0   & 17 & $-$101
    \end{tabular}
\end{table}

Dermed får vi en værdi for $R$ til at være 6. Dette eftertjekkes ved $6\cdot17 = 102 \equiv 1 \; \; (\text{mod} \; 101)$. Det fundne $R$ er altså en løsning til \eqref{eq:17d}.\\

Vi ganger med $R$ på begge sider af \eqref{eq:17d}:
\begin{equation}
17 dR \equiv 1R \; \; (\text{mod} \; 101)
\end{equation}
\begin{equation}
\Rightarrow d \equiv 6  \; \; (\text{mod} \; 101)
\end{equation}

Dette giver os følgende mulige værdier for $d$: 
\begin{gather}\label{eq:dLoesning}
    d = 6 + 101\mathds{Z}
\end{gather}

Ved at indsætte \eqref{eq:dLoesning} i ligning \eqref{eq:xd}, findes løsningsmængden for $x$:
\begin{gather*}
    x = 17d + 1 = 17(6 + 101\mathds{Z}) + 1
\end{gather*}

Dvs. løsningsmængden er restklassen
\begin{equation}
x = 103 + 1717\mathds{Z}
\end{equation}

\textbf{Kontrol}\\
Vi udfører kontrol af den fundne løsning.\\
Kontrol af \eqref{eq:2a1}: Vi tjekker om en $x$ minus 1 er et multiplum af 17. 
\begin{equation}
103 + 1717 = 1 + 17d
\end{equation}
\begin{equation}
\Rightarrow d = \frac{103+1717-1}{17} = 107
\end{equation}

Det udregnede $d$ er et heltal som forventet.\\
\\
Kontrol af \eqref{eq:2a2}: Vi tjekker om $x$ minus 2 er et multiplum af 101.
\begin{equation}
103 + 1717 = 2 + 101d
\end{equation}
\begin{equation}
\Rightarrow d = \frac{103+1717-2}{101} = 18
\end{equation}

Det udregnede $d$ er et heltal som forventet.

\subsection*{b)}
Vi ønsker at undersøge, om følgende ligningssystem har en ikke-tom løsningsmængde:
\begin{equation}\label{eq:A}
x \equiv 7 \text{ (mod 15)}
\end{equation}
\vspace{-0.55cm}
\begin{equation}\label{eq:B}
x \equiv 14 \text{  (mod 33)}    
\end{equation}

Vi ser at $\text{sfd}(15, 33) = 3$. Der er dermed ikke garanteret en løsning til ligningssystemet.

Prøver vi samme fremgangsmåde som i opgave 2 \textbf{a)}, får vi følgende ligninger, hvor $c,d \in \mathds{Z}$:
\begin{gather*}
    7 + 15c = 14 + 33d  \\
    \Rightarrow 33d = -7 + 15c
\end{gather*}
Ligningerne kan omskrives til kongruensligningen
\begin{gather*}
    33 d \equiv -7 \; \; (\text{mod} \; 15)
\end{gather*}

Vi skal dermed finde $R$ som opfylder $33R \equiv 1 \; \; (\text{mod} \; 15)$, svarende til den multiplikativ inverse af 33, modulo 15:

\begin{table}[H]
    \centering
    \begin{tabular}{c|c|c|c}
    $k$ & $r_k$ & $s$  & $R$  \\ \hline
    1& 33   & 1  & 0  \\
    2& 15   & 0 & 1  \\
    3& 3   & 1  & $-$2 \\
    4& 0    & $-$5 & 1
    \end{tabular}
\end{table}

Vi kan altså ikke finde et tal $R$ der opfylder $33R \equiv 1 \; \; (\text{mod} \; 15)$.\\
%\subsection*{b) Magnus}

%Vi vil undersøge om følgende system af kongruensligninger har en løsning:
%\begin{equation}\label{eq:A}
%x \equiv 7 \text{ (mod 15)}
%\end{equation}
%\vspace{-0.55cm}
%\begin{equation}\label{eq:B}
%x \equiv 14 \text{  (mod 33)}    
%\end{equation}

Vinket er at benytte den kinesiske restklassesætning i omvendt rækkefølge. Dvs. vi kan tænke på hver af ligningerne \eqref{eq:A} og \eqref{eq:B} som fremkommet ved at benytte den kinesiske restklassesætning (KRS).\\
\\
\textbf{Den kinesiske restklassesætning}\\
Når vi benytter den kinesiske restklassesætning, omskriver vi et system af kongruensligninger
\begin{equation}
x \equiv b_1 \text{  (mod } n_1), \quad b_1 \in \mathbb{Z}, \quad n_1 \in \mathbb{N}
\end{equation}
\vspace{-0.55cm}
\begin{equation}
x \equiv b_2 \text{  (mod } n_2), \quad b_2 \in \mathbb{Z}, \quad n_2 \in \mathbb{N}
\end{equation}

til én kongruensligning på formen
\begin{equation}\label{eq:KRS}
x \equiv u_1 n_1 b_2 + u_2 n_2 b_1 \text{ (mod } n_1 n_2)
\end{equation}

der overholder følgende krav:
\begin{equation}\label{eq:krav1}
\text{sfd}(n_1,n_2) = 1
\end{equation}
\vspace{-0.55cm}
\begin{equation}\label{eq:krav2}
u_1 n_1 + u_2 n_2 = 1, \quad u \in \mathbb{Z}
\end{equation}


\textbf{Første ligning:} $x \equiv 7 \text{ (mod 15)}$\\
Vi kræver at \eqref{eq:A} er fremkommet ved at anvende KRS. Vi sammenligner \eqref{eq:A} og kravet \eqref{eq:KRS}. Det ses at
\begin{equation}\label{eq:n1n2produkt1}
n_1 n_2 = 15
\end{equation}

For at benytte KRS, skal vi overholde kravet \eqref{eq:krav1}. En løsning, der overholder \eqref{eq:krav1} og \eqref{eq:n1n2produkt1}, er primtallene 3 og 5:
\begin{equation*}
n_1 = 3, \quad n_2 = 5
\end{equation*}

Vi bestemmer nu $u_1$ og $u_2$ så \eqref{eq:krav2} er overholdt. En løsning er 
\begin{equation*}
u_1 = 2, \quad u_2 = -1
\end{equation*}

da \begin{equation*}
u_1 n_1 + u_2 n_2 = 2 \cdot 3 + (-1) \cdot 5 = 1
\end{equation*}

Vi skal nu bestemme $b_1$ og $b_2$. For at ende med ligningen \eqref{eq:A} skal \eqref{eq:KRS} og \eqref{eq:A} have samme indmad til højre for kongruenstegnene. Dvs. 
\begin{equation*}
7 = u_1 n_1 b_2 + u_2 n_2 b_1
\end{equation*}

En mulig løsning er 
\begin{equation*}
b_1 = 1, \quad b_2 = 2
\end{equation*}

da
\begin{equation*}
u_1 n_1 b_2 + u_2 n_2 b_1 = 2 \cdot 3 \cdot 2 + (-1) \cdot 5 \cdot 1 = 12 - 5 = 7
\end{equation*}

Vi har nu omskrevet ligningen \eqref{eq:A} til systemet
\begin{equation}\label{eq:A1}
x \equiv 1 \text{ (mod 3)}
\end{equation}
\begin{equation}\label{eq:A2}
x \equiv 2 \text{  (mod 5)}    
\end{equation}


\textbf{Anden ligning:} $x \equiv 14 \text{ (mod 33)}$\\
Vi kræver at \eqref{eq:B} er fremkommet ved at anvende KRS. Vi sammenligner \eqref{eq:B} og kravet \eqref{eq:KRS}. Det ses at
\begin{equation}\label{eq:n1n2produkt2}
n_1 n_2 = 33
\end{equation}

For at benytte KRS, skal vi overholde kravet \eqref{eq:krav1}. En løsning, der overholder \eqref{eq:krav1} og \eqref{eq:n1n2produkt2}, er primtallene 3 og 11:
\begin{equation*}
n_1 = 3, \quad n_2 = 11
\end{equation*}

Vi bestemmer nu $u_1$ og $u_2$ så \eqref{eq:krav2} er overholdt. En løsning er 
\begin{equation*}
u_1 = 4, \quad u_2 = -1
\end{equation*}

da \begin{equation*}
u_1 n_1 + u_2 n_2 = 4 \cdot 3 + (-1) \cdot 11 = 1
\end{equation*}

Vi skal nu bestemme $b_1$ og $b_2$. For at ende med ligningen \eqref{eq:B} skal \eqref{eq:KRS} og \eqref{eq:B} have samme indmad til højre for kongruenstegnene. Dvs. 
\begin{equation*}
14 = u_1 n_1 b_2 + u_2 n_2 b_1
\end{equation*}

En mulig løsning er 
\begin{equation*}
b_1 = 2, \quad b_2 = 3
\end{equation*}

da
\begin{equation*}
u_1 n_1 b_2 + u_2 n_2 b_1 = 4 \cdot 3 \cdot 3 + (-1) \cdot 11 \cdot 2 = 36 - 22 = 14
\end{equation*}

Vi har nu omskrevet ligningen \eqref{eq:B} til systemet
\begin{equation}\label{eq:B1}
x \equiv 2 \text{ (mod 3)}
\end{equation}
\vspace{-0.55cm}
\begin{equation}\label{eq:B2}
x \equiv 3 \text{  (mod 11)}    
\end{equation}

\textbf{Løsningsmængde}\\
Det oprindelige ligningssystem \eqref{eq:A}-\eqref{eq:B} er ækvivalent med systemet givet ved de fire ligninger \eqref{eq:A1}, \eqref{eq:A2}, \eqref{eq:B1} og \eqref{eq:B2}.\\
\\
Antag nu at der eksisterer løsninger til det oprindelige system. Betragt nu de to kongruensligninger \eqref{eq:A1} og \eqref{eq:B1}:
\begin{equation}
x \equiv 1 \text{ (mod 3)} \quad \land \quad x \equiv 2 \text{ (mod 3)}
\end{equation}
Ved brug af lemma 3 fra noten:
Hvis $x \equiv 1  \text{ (mod 3)}$ og $x \equiv 2 \text{ (mod 3)}$, \\så er 
\begin{equation}\label{eq:modstrid}
1 \equiv 2 \text{ (mod 3)}
\end{equation}
Men \eqref{eq:modstrid} fører til en modstrid; 1 delt med 3 er 0 med 1 til rest. 2 delt med 3 er 0 med 2 til rest.\\
\\
Antagelsen om at der eksisterer løsninger fører til en modstrid, så det oprindelige ligningssystem \eqref{eq:A}-\eqref{eq:B} (og således også det omskrevne ligningssystem givet ved \eqref{eq:A1}, \eqref{eq:A2}, \eqref{eq:B1} og \eqref{eq:B2}) har en tom løsningsmængde.\\

I \texttt{Java} tjekker vi for heltal $c=0$ til $c=100000000$, om der eksisterer et heltal $d$ der opfylder $7 + 15c = 14 + 33d$. Dvs. for hvert $c$ isolerer vi $d$ og tjekker, om det er et heltal. Simuleringen giver ingen resultater, som forventet.


